\documentclass{resume} % Use the custom resume.cls style

\usepackage[left=0.4 in,top=0.4in,right=0.4 in,bottom=0.4in]{geometry} % Document margins
\usepackage{graphicx}
\newcommand{\tab}[1]{\hspace{.2667\textwidth}\rlap{#1}} 
\newcommand{\itab}[1]{\hspace{0 em}\rlap{#1}}
\name{ANKUJ UNIYAL} % Your name
% You can merge both of these into a single line, if you do not have a website.
\address{+91 8527613728 \\ Noida, India}
\address{\href{mail@mail.com} \\
\href{https://www.linkedin.com/in/<id>} \\ \href{https://github.com/ankujuniyal}
}
\begin{document}

%----------------------------------------------------------------------------------------
%	OBJECTIVE
%----------------------------------------------------------------------------------------

% \begin{rSection}{OBJECTIVE}

% Working as a DevOps engineer for 8 years, focusing on cloud infrastructure, CI/CD pipelines, and automation tools. I have a track record of improving efficiency and productivity by introducing solutions and fostering teamwork. I am deeply interested in transformation and excited to apply my expertise to drive change.


% \end{rSection}
%----------------------------------------------------------------------------------------
%	EDUCATION SECTION
%----------------------------------------------------------------------------------------

\begin{rSection}{Education}

%{\bf Master of Computer Science}, Stanford University \hfill {Expected 2020}\\
%Relevant Coursework: A, B, C, and D.

{\bf Bachelor of Technology I.T}, Uttarakhand Technical University, Dehradun \hfill {2009 - 2013}
%Minor in Linguistics \smallskip \\
%Member of Eta Kappa Nu \\
%Member of Upsilon Pi Epsilon \\


\end{rSection}

%----------------------------------------------------------------------------------------
% TECHINICAL STRENGTHS
%----------------------------------------------------------------------------------------
\begin{rSection}{SKILLS}

\begin{tabular}{ @{} >{\bfseries}l @{\hspace{4ex}} l }
Platform  & OpenShift Kuberne, GCP, CDN (Akamai), Kafka (Confluent),\\ &  DevSecOps (Blackduck, Coverity, Sonarqube), ServiceMesh, Api Getway (Kong),\\ & AWS, Docker, Git, JIRA\\
Web/App Servers & Apache Tomcat, Apache HTTP, Nginx, Weblogic\\
Automation & Jenkins, Jira worflow, Ansible, GoCD, Terrafrom\\
Monitoring & Prometheus, Grafana, Open telemetry, Elastic Stack (ELK), Zabbix\\
Programming Languages &  Python, Shell Scripting, Java (Basic)\\
Database &  Mongo, Mysql, influx, Elasticserach\\
Soft Skills & Interpersonal, Teamwork, Leadership\\
\end{tabular}\\
\end{rSection}

%----------------------------------------------------------------------------------------
%	Areas of Expertise
%----------------------------------------------------------------------------------------

% \begin{rSection}{Areas of Expertise}

% {OpenShift K8s • CNCF • Apache Kafka • Data Pipelines • Elastic Stack (ELK) • Git • Team Leadership • Google Cloud Platform (GCP) • DevSecOps • Blackduck • Coverity • Sonarqube  • ServiceMesh • CDN (Akamai)• Api Getway (Kong) • Helm • Mongodb • Maven • Python • Automation Tools (Ansible, Jenkins, GoCD ) • Monitoring Tools (Prometheus, Grafana, Zabbix ) }
% \end{rSection}

\begin{rSection}{EXPERIENCE}

\textbf{Sr. Devops Engineer } \hfill Jun 2021 - Present \\
Bharti Airtel Ltd. \hfill \textit{Gurugram, India}
 \begin{itemize}
    \itemsep -3pt {}
     \item Implementing Kafka data pipeline.
     \item Led the migration of middleware to Kubernetes.
     \item Implementing the CDN/GTM rule in Akamai.
     \item Managing API gateway(Kong)
    \item Enhanced CI/CD process to deploy applications on various platforms.
    % \item VRA REST API task automation.
 \end{itemize}

\textbf{Sr.Software Development Engineer} \hfill Jul 2017 - May 2021 (3 years 11 months)\\
Unify Technologies \hfill \textit{Gurugram, India}
 \begin{itemize}
    \itemsep -3pt {}
     \item Automated Zabbix Monitoring 100\% Infra using REST API, Python, and Flax skills.
     \item Led Kafka Deployment
    \item Integrated Kong API Gateway. Created customized plugins using Lua scripting to improve the functionality and efficiency of API's.
 \end{itemize}

\textbf{Linux System Administrator} \hfill Feb 2016 - Jan 2017 (1 year)\\
Silverpush \hfill \textit{Gurugram, India}
 % \begin{itemize}
 %    \itemsep -3pt {} 
 %     \item Automated Zabbix Monitoring 100\% Infra using REST API, Python, and Flax skills.
 %     \item Node based application 
 %    \item Implementing firewall policy on Sonicwall. 
 % \end{itemize}

 \textbf{System Administrator} \hfill Oct 2013 - Feb 2016 (2 years 5 months)\\
Rapidsoft Technologies \hfill \textit{Gurugram, India}

\end{rSection} 

%----------------------------------------------------------------------------------------
%	WORK EXPERIENCE SECTION
%----------------------------------------------------------------------------------------

% \begin{rSection}{PROJECTS}
% \vspace{-1.25em}
% \item \textbf{Hiring Search Tool.} {Built a tool to search for Hiring Managers and Recruiters by using ReactJS, NodeJS, Firebase and boolean queries. Over 25000 people have used it so far, with 5000+ queries being saved and shared, and search results even better than LinkedIn! \href{https://hiring-search.careerflow.ai/}{(Try it here)}}
% \item \textbf{Short Project Title.} {Build a project that does something and had quantified success using A, B, and C. This project's description spans two lines and also won an award.}
% \item \textbf{Short Project Title.} {Build a project that does something and had quantified success using A, B, and C. This project's description spans two lines and also won an award.}
% \end{rSection} 

%----------------------------------------------------------------------------------------
% \begin{rSection}{Extra-Curricular Activities} 
% \begin{itemize}
%     \item 	Actively write \href{https://www.faangpath.com/blog/}{blog posts} and social media posts (\href{https://www.tiktok.com/@faangpath}{TikTok}, \href{https://www.instagram.com/faangpath/?hl=en}{Instagram}) viewed by over 20K+ job seekers per week to help people with best practices to land their dream jobs. 
%     \item	Sample bullet point.
% \end{itemize}


% \end{rSection}

%----------------------------------------------------------------------------------------
\begin{rSection}{Licenses \& Certifications} 
\begin{itemize}
    \item RHCE - Red Hat \hfill 150-183-291
    \item RHCSA - Red Hat \hfill 150-183-291
    \item Learning Docker \hfill LinkedIn 
\end{itemize}


\end{rSection}


\end{document}
